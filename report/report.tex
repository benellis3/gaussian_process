\documentclass{article}

\usepackage[hmargin=20mm, vmargin=25mm]{geometry}
\usepackage{parskip}
\usepackage{amsmath}

\begin{document}
\title{Data, Estimation and Inference Report}
\author{Ben Ellis}
\date{\today}
\maketitle

% Describe the data that we are trying to fit and its patterns
This report considers data from a sensor measuring tide height.
The sensor often fails to transmit readings due to severe weather, and 
hence we here try to interpolate from the data available.

The readings from the sensor and the true tide heights over the period 
considered are shown in Figure [?]. 
As you can see this data exhibits a few interesting patterns:
\begin{itemize}
    \item As we would expect, the tide height is periodic with a period of roughly 12 hours.
    \item There are a few notable periods where data are entirely missing.
    \item Although data are missing, observed readings show relatively little noise.
\end{itemize}

The regular structure of the data suggests that we could use Gaussian Processes to interpolate.

\section*{Gaussian Processes}

A gaussian
% Why use gaussian process here? 
% Looks like a fairly regular function so limited expressivity not a problem
% Describe gaussian processes 
% Describe the marginal likelihood.
% Talk about 0 mean assumption and normalisation of data
% Talk about kernel choice. 



\end{document}

